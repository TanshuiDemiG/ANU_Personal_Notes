\documentclass{article}

% Language setting
% Replace `english' with e.g. `spanish' to change the document language
\usepackage[english]{babel}

% Set page size and margins
% Replace `letterpaper' with `a4paper' for UK/EU standard size
\usepackage[letterpaper,top=2cm,bottom=2cm,left=3cm,right=3cm,marginparwidth=1.75cm]{geometry}

\setlength{\parskip}{1em}
\usepackage{listings}

% Useful packages
\usepackage{amsmath}
\usepackage{graphicx}
\usepackage[colorlinks=true, allcolors=blue]{hyperref}
\usepackage{makecell, multirow, tabularx} % pack for table
\newcolumntype{C}{>{\centering\arraybackslash}X} % for centering
\usepackage{float}%for pic to follow words
\usepackage{ctex}% change abstract name
\ctexset{
abstractname = {Statement}
}

\title{MATH6005--Assignment 2}
\author{Chenxuan Sun}
\author{u7983176}
\begin{document}
	\maketitle
	\begin{abstract}
		The assignment I have submitted has been produced according to the rules on Page 1 of Assignment 1. The solutions produced have been written/typed on my own. I have not used AI to prepare my solutions. I have provided references of sources I have used when appropriate.
		\centering Student ID: u7983176
	\end{abstract}
	\newpage

	%%main

	% question 1.A
	\section{Question 1}
	\textbf{Part A}
	
	\textbf{I. Explain why the proof is incorrect.}
	
	The given proof attempts to show that the product of any even integer \( m \) and any odd integer \( n \) is even. \\
 	The proof is incorrect because it incorrectly assumes that both \( m \) and \( n \) can be expressed in terms of the same integer \( k \). Specifically, it assumes \( m = 2k \) and \( n = 2k + 1 \). This is not valid because \( m \) and \( n \) are independent integers, and there is no reason they should be related by the same \( k \).
	Here is a contradiction:\\
	If m = 4 =2k, , then k = 2. \\
	But substituting k =2 into the expression for n gives: \\
	\[n\=2k+1\=2(2)+1\=5n = 2k + 1 = 2(2) + 1 = 5n\=2k+1\=2(2)+1\=5\]
	
	II. Give a correct proof of the statement.
	
	Correct Proof:
	
	Let \( m \) be an even integer and \( n \) be an odd integer. By definition, there exists an integer \( a \) such that \( m = 2a \). Similarly, since \( n \) is odd, there exists an integer \( b \) such that \( n = 2b + 1 \).
	
	Consider the product \( mn \):
	\[ mn = (2a)(2b + 1) = 4ab + 2a = 2(2ab + a) \]
	
	Let \( q = 2ab + a \). Since \( a \) and \( b \) are integers, \( q \) is also an integer. Therefore, \( mn = 2q \), which by definition means \( mn \) is even.
	
	Thus, the product of any even integer and any odd integer is even.
	
	% question 1.B
	\textbf{Part B}
	
	Use mathematical induction to prove that for every integer \( n \geq 3 \),
	\[ \sum_{k=3}^{n} 4^k = 4(4^n - 16)/3 \]
	
	Proof by Induction:
	
	Base Case:
	
	For \( n = 3 \):
	\[ \sum_{k=3}^{3} 4^k = 4^3 = 64 \]
	\[ \frac{4(4^3 - 16)}{3} = \frac{4(64 - 16)}{3} = \frac{4 \cdot 48}{3} = 64 \]
	
	The base case holds.
	
	Inductive Step:
	
	Assume the statement is true for some integer \( n \geq 3 \). That is,
	\[ \sum_{k=3}^{n} 4^k = \frac{4(4^n - 16)}{3} \]
	
	We need to show that:
	\[ \sum_{k=3}^{n+1} 4^k = \frac{4(4^{n+1} - 16)}{3} \]
	
	Consider:
	\[ \sum_{k=3}^{n+1} 4^k = \sum_{k=3}^{n} 4^k + 4^{n+1} \]
	
	By the inductive hypothesis:
	\[ \sum_{k=3}^{n} 4^k = \frac{4(4^n - 16)}{3} \]
	
	Thus:
	\[ \sum_{k=3}^{n+1} 4^k = \frac{4(4^n - 16)}{3} + 4^{n+1} \]
	
	Factor out \( 4 \):
	\[ = \frac{4(4^n - 16) + 12 \cdot 4^{n+1}}{3} \]
	\[ = \frac{4(4^n - 16 + 3 \cdot 4^{n+1})}{3} \]
	\[ = \frac{4(4^{n+1} - 16)}{3} \]
	
	Thus, the statement holds for \( n+1 \).
	
	By the principle of mathematical induction, the statement is true for all integers \( n \geq 3 \).
	
	% question 1.C
	\textbf{Part C}	
	
	I. Trace the Selection Sort Algorithm on \( S \) with ordering rule "≥" and count how many comparisons were made.
	
	Selection Sort works by repeatedly finding the maximum element (considering ascending order) from the unsorted part and putting it at the beginning.
	
	Initial Array: S = [3, 1, 4, 1, 5, 9, 2, 6]
	
	1. Find the maximum element in \( S[0..7] \) and swap with \( S[0] \):
	- Comparisons: 7 (to find the max)
	- Array after step 1: [9, 1, 4, 1, 5, 3, 2, 6]
	
	2. Find the maximum element in \( S[1..7] \) and swap with \( S[1] \):
	- Comparisons: 6
	- Array after step 2: [9, 6, 4, 1, 5, 3, 2, 1]
	
	3. Find the maximum element in \( S[2..7] \) and swap with \( S[2] \):
	- Comparisons: 5
	- Array after step 3: [9, 6, 5, 1, 4, 3, 2, 1]
	
	4. Find the maximum element in \( S[3..7] \) and swap with \( S[3] \):
	- Comparisons: 4
	- Array after step 4: [9, 6, 5, 4, 1, 3, 2, 1]
	
	5. Find the maximum element in \( S[4..7] \) and swap with \( S[4] \):
	- Comparisons: 3
	- Array after step 5: [9, 6, 5, 4, 3, 1, 2, 1]
	
	6. Find the maximum element in \( S[5..7] \) and swap with \( S[5] \):
	- Comparisons: 2
	- Array after step 6: [9, 6, 5, 4, 3, 2, 1, 1]
	
	7. Find the maximum element in \( S[6..7] \) and swap with \( S[6] \):
	- Comparisons: 1
	- Array after step 7: [9, 6, 5, 4, 3, 2, 1, 1]
	
	Total comparisons: \( 7 + 6 + 5 + 4 + 3 + 2 + 1 = 28 \)
	
	II. Trace the Merge Sort Algorithm on \( S \) with ordering rule "≥" and count how many comparisons were made.
	
	Merge Sort works by dividing the array into halves, sorting each half, and then merging the sorted halves.
	
	Initial Array: S = [3, 1, 4, 1, 5, 9, 2, 6]
	
	1. Divide: [3, 1, 4, 1] and [5, 9, 2, 6]
	
	2. Divide: [3, 1] and [4, 1]; [5, 9] and [2, 6]
	
	3. Divide: [3] and [1]; [4] and [1]; [5] and [9]; [2] and [6]
	
	4. Merge: [1, 3]; [1, 4]; [5, 9]; [2, 6]
	- Comparisons: 1 + 1 + 1 + 1 = 4
	
	5. Merge: [1, 1, 3, 4]; [2, 5, 6, 9]
	- Comparisons: 3 + 3 = 6
	
	6. Merge: [1, 1, 2, 3, 4, 5, 6, 9]
	- Comparisons: 7
	
	Total comparisons: \( 4 + 6 + 7 = 17 \)
	
	III. Without tracing the algorithm, how many comparisons are required to complete the Selection Sort Algorithm on \( S \) with ordering rule "≤". Explain your answer.
	
	For Selection Sort, the number of comparisons required to sort an array of \( n \) elements is always:
	\[ \frac{n(n-1)}{2} \]
	
	For \( n = 8 \):
	\[ \frac{8(8-1)}{2} = \frac{8 \cdot 7}{2} = 28 \]
	
	IV. Explain why we would need to trace the Merge Sort Algorithm on \( S \) with ordering rule "≤" if we wanted to count the number of comparisons it required.
	
	Merge Sort's number of comparisons depends on the specific order of elements and how they are divided and merged. The number of comparisons in Merge Sort is not fixed for a given \( n \) but depends on the data itself. Therefore, to count the exact number of comparisons, we must trace the algorithm.
	
	\section{Question 2}
	\textbf{Part A}
	
	I. Linearity of the Function F
	
	Let \( F: \mathbb{Q}^3 \to \mathbb{Q}^3 \) be defined by
	
	\[ F(x, y, z) = (4x - 3y + 2z, 3y, x + y + z). \]
	
	We need to determine whether \( F \) is a linear function. A function \( F \) is linear if it satisfies the following properties for all vectors \( \mathbf{u}, \mathbf{v} \in \mathbb{Q}^3 \) and all scalars \( c \in \mathbb{Q} \):
	
	1. \( F(\mathbf{u} + \mathbf{v}) = F(\mathbf{u}) + F(\mathbf{v}) \)
	2. \( F(c\mathbf{u}) = cF(\mathbf{u}) \)
	
	Let's check these properties.
	
	1. Additivity:
	
	Let \( \mathbf{u} = (x_1, y_1, z_1) \) and \( \mathbf{v} = (x_2, y_2, z_2) \).
	
	\[ \mathbf{u} + \mathbf{v} = (x_1 + x_2, y_1 + y_2, z_1 + z_2). \]
	
	Now, compute \( F(\mathbf{u} + \mathbf{v}) \):
	
	\[ F(\mathbf{u} + \mathbf{v}) = F(x_1 + x_2, y_1 + y_2, z_1 + z_2) = (4(x_1 + x_2) - 3(y_1 + y_2) + 2(z_1 + z_2), 3(y_1 + y_2), (x_1 + x_2) + (y_1 + y_2) + (z_1 + z_2)). \]
	
	Simplify the components:
	
	\[ F(\mathbf{u} + \mathbf{v}) = (4x_1 + 4x_2 - 3y_1 - 3y_2 + 2z_1 + 2z_2, 3y_1 + 3y_2, x_1 + x_2 + y_1 + y_2 + z_1 + z_2). \]
	
	Now, compute \( F(\mathbf{u}) + F(\mathbf{v}) \):
	
	\[ F(\mathbf{u}) = (4x_1 - 3y_1 + 2z_1, 3y_1, x_1 + y_1 + z_1), \]
	\[ F(\mathbf{v}) = (4x_2 - 3y_2 + 2z_2, 3y_2, x_2 + y_2 + z_2). \]
	
	Add these two vectors:
	
	\[ F(\mathbf{u}) + F(\mathbf{v}) = (4x_1 - 3y_1 + 2z_1 + 4x_2 - 3y_2 + 2z_2, 3y_1 + 3y_2, x_1 + y_1 + z_1 + x_2 + y_2 + z_2). \]
	
	We see that:
	
	\[ F(\mathbf{u} + \mathbf{v}) = F(\mathbf{u}) + F(\mathbf{v}). \]
	
	2. Homogeneity:
	
	Let \( c \in \mathbb{Q} \) and \( \mathbf{u} = (x, y, z) \).
	
	\[ c\mathbf{u} = (cx, cy, cz). \]
	
	Now, compute \( F(c\mathbf{u}) \):
	
	\[ F(c\mathbf{u}) = F(cx, cy, cz) = (4(cx) - 3(cy) + 2(cz), 3(cy), cx + cy + cz). \]
	
	Simplify the components:
	
	\[ F(c\mathbf{u}) = (4cx - 3cy + 2cz, 3cy, cx + cy + cz). \]
	
	Now, compute \( cF(\mathbf{u}) \):
	
	\[ F(\mathbf{u}) = (4x - 3y + 2z, 3y, x + y + z), \]
	
	\[ cF(\mathbf{u}) = c(4x - 3y + 2z, 3y, x + y + z) = (c(4x - 3y + 2z), c(3y), c(x + y + z)). \]
	
	We see that:
	
	\[ F(c\mathbf{u}) = cF(\mathbf{u}). \]
	
	Since both properties are satisfied, \( F \) is a linear function.
	
	II. Linearity of the Function f
	
	Let \( f: \mathbb{Q}^3 \to \mathbb{Q}^3 \) be defined by
	
	\[ f(x, y, z) = (4x - 3y + 2z, 3y + 1, x + y). \]
	
	We need to determine whether \( f \) is a linear function. A function \( f \) is linear if it satisfies the following properties for all vectors \( \mathbf{u}, \mathbf{v} \in \mathbb{Q}^3 \) and all scalars \( c \in \mathbb{Q} \):
	
	1. \( f(\mathbf{u} + \mathbf{v}) = f(\mathbf{u}) + f(\mathbf{v}) \)
	2. \( f(c\mathbf{u}) = c f(\mathbf{u}) \)
	
	Let's check these properties.
	
	1. Additivity:
	
	Let \( \mathbf{u} = (x_1, y_1, z_1) \) and \( \mathbf{v} = (x_2, y_2, z_2) \).
	
	\[ \mathbf{u} + \mathbf{v} = (x_1 + x_2, y_1 + y_2, z_1 + z_2). \]
	
	Now, compute \( f(\mathbf{u} + \mathbf{v}) \):
	
	\[ f(\mathbf{u} + \mathbf{v}) = f(x_1 + x_2, y_1 + y_2, z_1 + z_2) = (4(x_1 + x_2) - 3(y_1 + y_2) + 2(z_1 + z_2), 3(y_1 + y_2) + 1, (x_1 + x_2) + (y_1 + y_2)). \]
	
	Simplify the components:
	
	\[ f(\mathbf{u} + \mathbf{v}) = (4x_1 + 4x_2 - 3y_1 - 3y_2 + 2z_1 + 2z_2, 3y_1 + 3y_2 + 1, x_1 + x_2 + y_1 + y_2). \]
	
	Now, compute \( f(\mathbf{u}) + f(\mathbf{v}) \):
	
	\[ f(\mathbf{u}) = (4x_1 - 3y_1 + 2z_1, 3y_1 + 1, x_1 + y_1), \]
	\[ f(\mathbf{v}) = (4x_2 - 3y_2 + 2z_2, 3y_2 + 1, x_2 + y_2). \]
	
	Add these two vectors:
	
	\[ f(\mathbf{u}) + f(\mathbf{v}) = (4x_1 - 3y_1 + 2z_1 + 4x_2 - 3y_2 + 2z_2, 3y_1 + 1 + 3y_2 + 1, x_1 + y_1 + x_2 + y_2). \]
	
	\[ f(\mathbf{u}) + f(\mathbf{v}) = (4x_1 - 3y_1 + 2z_1 + 4x_2 - 3y_2 + 2z_2, 3y_1 + 3y_2 + 2, x_1 + x_2 + y_1 + y_2). \]
	
	We see that:
	
	\[ f(\mathbf{u} + \mathbf{v}) \neq f(\mathbf{u}) + f(\mathbf{v}). \]
	
	Since the additivity property is not satisfied, \( f \) is not a linear function.
	
	\textbf{Part B}

	
	Let \( \{ s_n \}_{n \in \mathbb{N}} \subseteq \mathbb{Q}^2 \), where \( s_n = (a_n, b_n) \) is defined implicitly by:
	
	\[ s_1 = (1, 0) \]
	\[ \forall n \in \mathbb{N} \left( a_{n+1} = 77a_n - 90b_n, \ b_{n+1} = 60a_n - 70b_n \right). \]
	
	I. Find the 2 × 2 matrix X such that \( \begin{pmatrix} a_{n+1} \\ b_{n+1} \end{pmatrix} = X \begin{pmatrix} a_n \\ b_n \end{pmatrix} \)
	
	From the recurrence relations, we can write:
	
	\[ \begin{pmatrix} a_{n+1} \\ b_{n+1} \end{pmatrix} = \begin{pmatrix} 77 & -90 \\ 60 & -70 \end{pmatrix} \begin{pmatrix} a_n \\ b_n \end{pmatrix}. \]
	
	Thus, the matrix \( X \) is:
	
	\[ X = \begin{pmatrix} 77 & -90 \\ 60 & -70 \end{pmatrix}. \]
	
	II. Calculate \( \det(A) \) and \( A^{-1} \)
	
	Let \( A = \begin{pmatrix} -6 & 5 \\ -5 & 4 \end{pmatrix} \).
	
	The determinant of \( A \) is given by:
	
	\[ \det(A) = (-6)(4) - (5)(-5) = -24 + 25 = 1. \]
	
	To find the inverse of \( A \), we use the formula for the inverse of a 2x2 matrix:
	
	\[ A^{-1} = \frac{1}{\det(A)} \begin{pmatrix} d & -b \\ -c & a \end{pmatrix} = \begin{pmatrix} 4 & -5 \\ 5 & -6 \end{pmatrix}. \]
	
	III. Show that \( X = A \begin{pmatrix} 2 & 0 \\ 0 & 5 \end{pmatrix} A^{-1} \)
	
	We need to verify the equality:
	
	\[ X = A \begin{pmatrix} 2 & 0 \\ 0 & 5 \end{pmatrix} A^{-1}. \]
	
	Compute the right-hand side:
	
	\[ A \begin{pmatrix} 2 & 0 \\ 0 & 5 \end{pmatrix} = \begin{pmatrix} -6 & 5 \\ -5 & 4 \end{pmatrix} \begin{pmatrix} 2 & 0 \\ 0 & 5 \end{pmatrix} = \begin{pmatrix} -12 & 25 \\ -10 & 20 \end{pmatrix}. \]
	
	Now, multiply by \( A^{-1} \):
	
	\[ \begin{pmatrix} -12 & 25 \\ -10 & 20 \end{pmatrix} \begin{pmatrix} 4 & -5 \\ 5 & -6 \end{pmatrix} = \begin{pmatrix} 4(-12) + 25(5) & -5(-12) + 25(-6) \\ 4(-10) + 20(5) & -5(-10) + 20(-6) \end{pmatrix} \]
	
	\[ = \begin{pmatrix} -48 + 125 & 60 - 150 \\ -40 + 100 & 50 - 120 \end{pmatrix} = \begin{pmatrix} 77 & -90 \\ 60 & -70 \end{pmatrix}. \]
	
	Thus, \( X = A \begin{pmatrix} 2 & 0 \\ 0 & 5 \end{pmatrix} A^{-1} \).
	
	IV. Find \( s_{101} \) using parts (I)-(III)
	
	Given \( s_1 = (1, 0) \) and the recurrence relation, we can write:
	
	\[ s_{101} = X^{100} s_1. \]
	
	From part (III), we have:
	
	\[ X = A \begin{pmatrix} 2 & 0 \\ 0 & 5 \end{pmatrix} A^{-1}. \]
	
	Thus,
	
	\[ X^{100} = \left( A \begin{pmatrix} 2 & 0 \\ 0 & 5 \end{pmatrix} A^{-1} \right)^{100} = A \begin{pmatrix} 2^{100} & 0 \\ 0 & 5^{100} \end{pmatrix} A^{-1}. \]
	
	Now, compute \( X^{100} s_1 \):
	
	\[ X^{100} s_1 = A \begin{pmatrix} 2^{100} & 0 \\ 0 & 5^{100} \end{pmatrix} A^{-1} \begin{pmatrix} 1 \\ 0 \end{pmatrix}. \]
	
	\[ = A \begin{pmatrix} 2^{100} & 0 \\ 0 & 5^{100} \end{pmatrix} \begin{pmatrix} 4 \\ 5 \end{pmatrix}. \]
	
	\[ = A \begin{pmatrix} 2^{100} \cdot 4 \\ 5^{100} \cdot 5 \end{pmatrix} = A \begin{pmatrix} 2^{100} \cdot 4 \\ 5^{100} \cdot 5 \end{pmatrix}. \]
	
	\[ = A \begin{pmatrix} 4 \cdot 2^{100} \\ 5 \cdot 5^{100} \end{pmatrix}. \]
	
	\[ = \begin{pmatrix} -6 & 5 \\ -5 & 4 \end{pmatrix} \begin{pmatrix} 4 \cdot 2^{100} \\ 5 \cdot 5^{100} \end{pmatrix}. \]
	
	\[ = \begin{pmatrix} -6(4 \cdot 2^{100}) + 5(5 \cdot 5^{100}) \\ -5(4 \cdot 2^{100}) + 4(5 \cdot 5^{100}) \end{pmatrix}. \]
	
	\[ = \begin{pmatrix} -24 \cdot 2^{100} + 25 \cdot 5^{100} \\ -20 \cdot 2^{100} + 20 \cdot 5^{100} \end{pmatrix}. \]
	
	\[ = \begin{pmatrix} 25 \cdot 5^{100} - 24 \cdot 2^{100} \\ 20 \cdot (5^{100} - 2^{100}) \end{pmatrix}. \]
	
	Thus, \( s_{101} \) is:
	
	\[ s_{101} = \left( 25 \cdot 5^{100} - 24 \cdot 2^{100}, 20 \cdot (5^{100} - 2^{100}) \right). \]
	
	
	\section{Question 3}
	\textbf{Part A}
	Calculate \( S(4, 3) \) by finding all possible partitions of \(\{1,2,3,4\}\) into 3 disjoint non-empty subsets.
	
	To find \( S(4, 3) \), we need to determine all possible ways to partition the set \(\{1, 2, 3, 4\}\) into 3 non-empty subsets. 
	
	Let's list all possible partitions:
	
	1. \(\{\{1\}, \{2\}, \{3, 4\}\}\)
	2. \(\{\{1\}, \{3\}, \{2, 4\}\}\)
	3. \(\{\{1\}, \{4\}, \{2, 3\}\}\)
	4. \(\{\{2\}, \{3\}, \{1, 4\}\}\)
	5. \(\{\{2\}, \{4\}, \{1, 3\}\}\)
	6. \(\{\{3\}, \{4\}, \{1, 2\}\}\)
	
	Thus, there are 6 ways to partition the set \(\{1, 2, 3, 4\}\) into 3 non-empty subsets, so \( S(4, 3) = 6 \).
	
	\textbf{Part B}
	Prove that \( S(n, k) = S(n-1, k-1) + kS(n-1, k) \) for \( 0 < k < n \).
	
	Consider all possible partitions of the set \(\{1, 2, ..., n\}\) into \( k \) disjoint non-empty subsets. We can categorize these partitions based on whether the element \( n \) is in its own subset or not.
	
	1. If \( n \) is in its own subset, then we need to partition the remaining \( n-1 \) elements into \( k-1 \) non-empty subsets. The number of such partitions is \( S(n-1, k-1) \).
	
	2. If \( n \) is not in its own subset, then \( n \) must be in one of the \( k \) subsets formed by the remaining \( n-1 \) elements. The number of ways to partition \( n-1 \) elements into \( k \) subsets is \( S(n-1, k) \), and for each of these partitions, there are \( k \) choices for which subset to add \( n \) to. Thus, the number of such partitions is \( kS(n-1, k) \).
	
	Combining these two cases, we get:
	\[ S(n, k) = S(n-1, k-1) + kS(n-1, k) \]
	
	\textbf{Part C}
	Prove that the number of surjective functions from \( X \) to \( Y \) is \( k! S(n, k) \).
	
	A surjective function \( f: X \to Y \) maps each element of \( X \) to some element of \( Y \) such that every element of \( Y \) is mapped to by at least one element of \( X \). 
	
	Given a surjective function \( f \), it induces a partition of \( X \) into \( k \) non-empty subsets, where each subset corresponds to the preimage of an element in \( Y \). The number of ways to partition \( X \) into \( k \) non-empty subsets is \( S(n, k) \).
	
	For each of these partitions, we need to assign each of the \( k \) subsets to one of the \( k \) elements in \( Y \). There are \( k! \) ways to do this assignment.
	
	Thus, the total number of surjective functions from \( X \) to \( Y \) is:
	\[ k! S(n, k) \]
	
	\textbf{Part D}
	Suppose \( S(4,2) = 7 \). When \( n = 5 \) and \( k = 3 \), calculate the probability that the function generated by the machine is surjective.
	
	First, we need to calculate the number of surjective functions from a 5-element set \( X \) to a 3-element set \( Y \). Using the formula from part C, the number of surjective functions is:
	\[ 3! S(5, 3) \]
	
	Using the recurrence relation from part B, we can calculate \( S(5, 3) \):
	\[ S(5, 3) = S(4, 2) + 3S(4, 3) \]
	Given \( S(4, 2) = 7 \) and \( S(4, 3) = 6 \), we have:
	\[ S(5, 3) = 7 + 3 \cdot 6 = 7 + 18 = 25 \]
	
	So, the number of surjective functions is:
	\[ 3! S(5, 3) = 6 \cdot 25 = 150 \]
	
	The total number of functions from \( X \) to \( Y \) is \( 3^5 = 243 \).
	
	Thus, the probability that a randomly generated function is surjective is:
	\[ \frac{150}{243} = \frac{50}{81} \approx 0.6173 \]
	
	Therefore, the probability that the function generated by the machine is surjective is approximately \( 0.6173 \) or \( \frac{50}{81} \).
	
	
	
	\section{Question 4}
	\textbf{Part A}
	
	A byte is a binary string of 8 bits. We need to find how many bytes contain at least five 1's and one 0.
	
	First, let's calculate the total number of bytes that contain exactly \( k \) ones, where \( k \) ranges from 5 to 8.
	
	The total number of 8-bit binary strings is \( 2^8 = 256 \).
	
	To find the number of bytes with exactly \( k \) ones, we use the binomial coefficient:
	
	\[
	\binom{8}{k}
	\]
	
	We need to sum this for \( k = 5, 6, 7, 8 \):
	
	\[
	\sum_{k=5}^{8} \binom{8}{k}
	\]
	
	Let's calculate each term separately:
	
	\[
	\binom{8}{5} = \frac{8!}{5!(8-5)!} = \frac{8!}{5!3!} = 56
	\]
	
	\[
	\binom{8}{6} = \frac{8!}{6!(8-6)!} = \frac{8!}{6!2!} = 28
	\]
	
	\[
	\binom{8}{7} = \frac{8!}{7!(8-7)!} = \frac{8!}{7!1!} = 8
	\]
	
	\[
	\binom{8}{8} = \frac{8!}{8!(8-8)!} = \frac{8!}{8!0!} = 1
	\]
	
	Adding these together:
	
	\[
	\binom{8}{5} + \binom{8}{6} + \binom{8}{7} + \binom{8}{8} = 56 + 28 + 8 + 1 = 93
	\]
	
	Therefore, the number of bytes that contain at least five 1's and one 0 is 93.
	
	\textbf{Part B}
	
	The bank requires a password for you to access your account. The password is a sequence of 12 characters that contains exactly:
	- 4 lower case alphabetical letters,
	- 2 upper case alphabetical letters,
	- 4 decimal digits, and
	- 2 special characters from the set \{ @, #, $, % \}.
	
	We need to calculate the total number of possible passwords, denoted by \( |X| \).
	
	1. Choosing positions for each type of character:
	- Choose 4 positions out of 12 for lower case letters:
	\[
	\binom{12}{4}
	\]
	- Choose 2 positions out of the remaining 8 for upper case letters:
	\[
	\binom{8}{2}
	\]
	- Choose 4 positions out of the remaining 6 for digits:
	\[
	\binom{6}{4}
	\]
	- The remaining 2 positions are for special characters.
	
	2. Calculating the number of ways to choose the characters:
	- There are 26 lower case letters, 26 upper case letters, 10 digits, and 4 special characters.
	- The number of ways to choose 4 lower case letters:
	\[
	26^4
	\]
	- The number of ways to choose 2 upper case letters:
	\[
	26^2
	\]
	- The number of ways to choose 4 digits:
	\[
	10^4
	\]
	- The number of ways to choose 2 special characters:
	\[
	4^2
	\]
	
	Combining these, the total number of possible passwords is:
	
	\[
	|X| = \binom{12}{4} \cdot \binom{8}{2} \cdot \binom{6}{4} \cdot 26^4 \cdot 26^2 \cdot 10^4 \cdot 4^2
	\]
	
	Calculating the binomial coefficients:
	
	\[
	\binom{12}{4} = \frac{12!}{4!(12-4)!} = 495
	\]
	
	\[
	\binom{8}{2} = \frac{8!}{2!(8-2)!} = 28
	\]
	
	\[
	\binom{6}{4} = \frac{6!}{4!(6-4)!} = 15
	\]
	
	So,
	
	\[
	|X| = 495 \cdot 28 \cdot 15 \cdot 26^6 \cdot 10^4 \cdot 4^2
	\]
	
	Now we need to find the least value of \( k \) such that the clash score \( N \) is no more than \( 10^{15} \).
	
	The clash score \( N \) is given by:
	
	\[
	N = \max_{y \in Y} | \{ x \in X | f(x) = y \} |
	\]
	
	We need \( N \leq 10^{15} \).
	
	Since \( Y \) is the set of all \( k \)-digit PINs, \( |Y| = 10^k \).
	
	We need:
	
	\[
	\frac{|X|}{|Y|} \leq 10^{15}
	\]
	
	Thus,
	
	\[
	\frac{495 \cdot 28 \cdot 15 \cdot 26^6 \cdot 10^4 \cdot 4^2}{10^k} \leq 10^{15}
	\]
	
	\[
	495 \cdot 28 \cdot 15 \cdot 26^6 \cdot 10^4 \cdot 4^2 \leq 10^{15+k}
	\]
	
	Taking logarithms (base 10) on both sides:
	
	\[
	\log_{10}(495) + \log_{10}(28) + \log_{10}(15) + 6 \log_{10}(26) + 4 + 2 \log_{10}(4) \leq 15 + k
	\]
	
	Approximating:
	
	\[
	\log_{10}(495) \approx 2.7, \quad \log_{10}(28) \approx 1.45, \quad \log_{10}(15) \approx 1.18
	\]
	
	\[
	\log_{10}(26) \approx 1.41, \quad \log_{10}(4) \approx 0.6
	\]
	
	\[
	2.7 + 1.45 + 1.18 + 6 \cdot 1.41 + 4 + 2 \cdot 0.6 \leq 15 + k
	\]
	
	\[
	2.7 + 1.45 + 1.18 + 8.46 + 4 + 1.2 \leq 15 + k
	\]
	
	\[
	18.99 \leq 15 + k
	\]
	
	\[
	k \geq 3.99
	\]
	
	Since \( k \) must be an integer, the least value of \( k \) is 4.
	
	Thus, the least value of \( k \) that may give a clash score of no more than \( 10^{15} \) is 4.
	%%
	
	
	
	


	
\end{document}
